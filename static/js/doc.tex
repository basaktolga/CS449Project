
\documentclass[12pt]{article}
\usepackage[a4paper,margin=1in]{geometry}
\usepackage{listings}
\usepackage{xcolor}
\usepackage{hyperref}

\definecolor{lightgray}{gray}{0.95}
\lstset{
  backgroundcolor=\color{lightgray},
  basicstyle=\ttfamily\small,
  keywordstyle=\color{blue}\bfseries,
  commentstyle=\color{green!60!black},
  stringstyle=\color{red},
  showstringspaces=false,
  breaklines=true,
  frame=single,
  numbers=left,
  numberstyle=\tiny,
  xleftmargin=15pt,
}

\title{Gesture-Based Interaction System Using MediaPipe}
\author{Doruk Dülger, Ulaş Meriç, Tolga Başak}
\date{\today}

\begin{document}

\maketitle

\section{Overview of the System}
This project implements a gesture-based interaction system using \textbf{MediaPipe} for real-time smile detection. The system is integrated into a web-based login and registration platform to enhance user security through a smile-based CAPTCHA verification.

Additionally, the system implements a gesture-based logout mechanism that combines facial expressions (right wink) with hand gestures (open palm hold) to provide a secure and intuitive way to log out of the system.

\subsection{Objective}
The objective is to detect a smile using facial landmarks and validate the user's action by triggering a CAPTCHA success event. The interaction is visually confirmed on the interface.

\subsection{Gesture Chosen}
\textbf{Smile Detection:} A facial gesture detected using key landmarks around the mouth, calculated in real-time.

\subsection{Landmarks Used}
\begin{itemize}
    \item Upper lip center (landmark 13)
    \item Lower lip center (landmark 14)
    \item Left mouth corner (landmark 61)
    \item Right mouth corner (landmark 291)
\end{itemize}

For the logout gesture detection:
\begin{itemize}
    \item Right eye landmarks (159 for upper, 145 for lower)
    \item Hand landmarks for palm detection:
        \begin{itemize}
            \item Palm base (landmark 0)
            \item Thumb tip (landmark 4)
            \item Index tip (landmark 8)
            \item Middle tip (landmark 12)
            \item Ring tip (landmark 16)
            \item Pinky tip (landmark 20)
        \end{itemize}
\end{itemize}

\subsection{Interaction Triggered}
When a valid smile is detected:
\begin{itemize}
    \item The CAPTCHA is marked as solved.
    \item A success message is displayed, confirming the action.
\end{itemize}

\subsection{Gesture-Based Logout:} A combination of right eye wink and open palm hold for 1.5 seconds triggers the logout sequence.

\section{Implementation Details}
This section explains how the system works, including the calculation of smile gestures and code implementations.

\subsection{Integration of MediaPipe}
The \textbf{MediaPipe Face Mesh} library is used for real-time detection of facial landmarks. The video feed is processed frame-by-frame, and specific mouth-related landmarks are analyzed to determine if the user is smiling.

\subsection{Landmark Calculation}
The detection logic is based on:
\begin{itemize}
    \item \textbf{Smile Ratio:} Calculated as:
    \[
    \text{Smile Ratio} = \frac{\text{Mouth Width}}{\text{Vertical Opening}}
    \]
    \item \textbf{Corner Lift  :} Measured as the distance between the upper lip center and the average height of the mouth corners.
\end{itemize}

If the smile ratio exceeds 3.3 and the corner lift is greater than 0.015, a smile is detected.

\subsection{Code Explanation}
\textbf{Login Template (login.html):} This Django template includes a smile-based CAPTCHA and fluid simulation background.

\begin{lstlisting}[language=HTML, caption=Login Template (login.html)]
<canvas id="fluid-canvas"></canvas>

<div class="relative z-10 max-w-4xl mx-auto mt-40 p-8 bg-white rounded-lg">
    <h2 class="text-2xl font-bold mb-5 text-center">Login</h2>
    <form method="post" onsubmit="return validateForm(event)">
        
        <label for="username">Username</label>
        <input type="text" id="username" name="username" required>
        <label for="password">Password</label>
        <input type="password" id="password" name="password" required>
        <input type="hidden" id="captcha-solved" name="captcha-solved" value="false">
        <button type="submit">Login</button>
    </form>
</div>
<script src=""></script>
\end{lstlisting}

\textbf{JavaScript for CAPTCHA Detection:} This script processes the video feed and identifies smiles using MediaPipe landmarks.

\begin{lstlisting}[language=JavaScript, caption=CAPTCHA Detection (face\_captcha.js)]
function setupFaceCaptcha() {
    const videoElement = document.createElement('video');
    const canvasElement = document.createElement('canvas');
    const faceMesh = new FaceMesh({
        locateFile: (file) => `https://cdn.jsdelivr.net/npm/@mediapipe/face_mesh/${file}`
    });

    faceMesh.onResults((results) => {
        const landmarks = results.multiFaceLandmarks[0];
        const upperLipCenter = landmarks[13];
        const lowerLipCenter = landmarks[14];
        const leftMouthCorner = landmarks[61];
        const rightMouthCorner = landmarks[291];

        const smileRatio = Math.abs(leftMouthCorner.x - rightMouthCorner.x) /
                          Math.abs(upperLipCenter.y - lowerLipCenter.y);

        if (smileRatio > 3.3) {
            document.getElementById('captcha-solved').value = 'true';
            alert('Smile detected!');
        }
    });

    new Camera(videoElement, { onFrame: async () => await faceMesh.send({ image: videoElement }) }).start();
}
\end{lstlisting}

\textbf{Fluid Simulation Background:} This enhances the interface visually.

\begin{lstlisting}[language=JavaScript, caption=Fluid Simulation]
<script src=""></script>
<style>
    #fluid-canvas {
        position: fixed;
        top: 0; left: 0;
        width: 100%; height: 100%;
        z-index: 0;
    }
</style>
\end{lstlisting}

\section{Group Contributions}
\begin{itemize}
    \item \textbf{Doruk Dülger:} Developed the smile detection algorithm and integrated MediaPipe.
    \item \textbf{Ulaş Meriç:} Designed the fluid simulation background and enhanced the UI.
    \item \textbf{Tolga Başak:} Worked on backend integration and security features.
\end{itemize}

\section{Video Demonstration}
A short video demonstrating the system in action can be accessed at: \href{https://your-video-link.com}{Demo Video}.

\section{GitHub Repository}
The project repository is available at: \href{https://github.com/your-repo-link}{GitHub Repository}. Commit history reflects contributions from all members.

\end{document}
